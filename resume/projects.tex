\cvsection{Academic Projects}
\begin{cventries}
	\cventry
	{Prof. Thies Justus} % Empty position
	{Articulation aware Canonical Surface Mapping} % Project
	{TUM Munich, Germany} % Empty location
	{April 2020 - July 2020} % Empty date
	{
		\begin{cvitems} % Description(s) bullet points
			\item {Implemened the algorithm from the
			\href{https://arxiv.org/abs/2004.00614}{\textbf{CSM} paper} to predict canonical
			surface mapping to map pixels of an object from an image to a 3D
			mesh template. Trained the model on \textbf{ImageNet (Zebra), p3d (car) and
			cub (bird)} datasets}
			\item {Implemened \textbf{U-Net} architecture to predict the surface
			mappings and \textbf{ResNet-50} powered model to predict camera parameters}
			\item {Used the latest \textbf{PyTorch} and \textbf{PyTorch3D} libraries for the rendering of 3D template}
		\end{cvitems}
	}

	\cventry
	{Vladimir Golkov, PhD Student} % Empty position
	{Visualizing and understanding Network Topologies} % Project
	{TUM Munich, Germany} % Empty location
	{Nov 2019 - April 2020} % Empty date
	{
		\begin{cvitems} % Description(s) bullet points
			\item {Helped designing a visual language for neural network topologies which is simpler, easier to understand and compare with other topologies}
			\item {Designed multiple famous network topologies \textbf{GAN,
			Auto Encoders} using the visual language.}
		\end{cvitems}
	}
	
	\cventry
	{Zahedi Ata, M.Sc.} % Empty position
	{Machine Learning on Building data} % Project
	{TUM Munich, Germany} % Empty location
	{June 2019 - Aug 2019} % Empty date
	{
		\begin{cvitems} % Description(s) bullet points
			\item {Using \textbf{Dynamo} extracted multiple features such of the area of the walls, distribution of area in different directions, from a building model imported into \textbf{Revit}.} 
			\item {Trained a machine learning model to predict the type of the building using the extracted features}
		\end{cvitems}
	}

  	\cventry
    {Prof. Thies Justus} % Empty position
    {3D Reconstruction via Direct Semi-Dense Visual Odometry using Stereo Camera} % Project
    {TUM Munich, Germany} % Empty location
    {Jan 2019 - Feb 2019} % Empty date
    {
      \begin{cvitems} % Description(s) bullet points
      	\item {Helped implementing an algorithm to estimate Odometry from stereo camera setup in real time at 30fps.}
		\item {Implemented a \textbf{block matching} algorithm to generate depth from stereo images.}
      \end{cvitems}
	}
	
	\cventry
	{Prof. Thies Justus} % Empty position
	{3D Scanning and Motion Capture} % Project
	{TUM Munich, Germany} % Empty location
	{Nov 2018 - Jan 2019} % Empty date
	{
		\begin{cvitems} % Description(s) bullet points
			\item {Implemented \textbf{Iterative Closest Point(ICP)} algorithm in c++ to align two bunny point clouds}
			\item {Implemented a simpler version of the bundle adjustment to \textbf{reconstruct a 3D scene}}
		\end{cvitems}
	}

	\cventry
	{Prof. Dr. Laura Leal-Taixé and Prof. Dr. Matthias Niessner} % Empty position
	{Key point prediction on face using CNN} % Project
	{TUM Munich, Germany} % Empty location
	{Jan 2018} % Empty date
	{
		\begin{cvitems} % Description(s) bullet points
			\item {Implemented a \textbf{Fully Convolutional Neural Network} to detect keypoints on a face using \textbf{PyTorch}.}
		\end{cvitems}
	}

	\cventry
	{Prof. Bhaskaran Raman and Prof. Kameshwari} % Empty position
	{SAFE (QuizApp)} % Project
	{IIT Mumbai, India} % Empty location
	{} % Empty date
	{
		\begin{cvitems} % Description(s) bullet points
			\item {Developed an \textbf{iOS} application in \textbf{swift} and Objective-C used by
			instructors to conduct online examinations in classroom for students
			securely}
		\end{cvitems}
	}
	% \cventry
	% {Prof. Parag Chaudari} % Empty position
	% {Star Wars droids animation} % Project
	% {IIT Mumbai, India} % Empty location
	% {} % Empty date
	% {
	% 	\begin{cvitems} % Description(s) bullet points
	% 		\item {Implemented a hierarchical model of two famous star-wars droids R2D2 and pit droid using \textbf{OpenGL} and \textbf{GLFW} and created an environment to put droids in and made an animation using them}
	% 	\end{cvitems}
	% }

\end{cventries}
